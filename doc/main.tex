
\documentclass[a4paper,12pt]{article}
\usepackage[czech,slovak]{babel}
\usepackage[a4paper, total={170mm, 240mm}, top=2cm, left=3cm, right=3cm]{geometry}
\usepackage[utf8]{inputenc}
\usepackage[IL2]{fontenc}
\usepackage{times}
\usepackage[hidelinks]{hyperref}
\usepackage{graphicx}
\usepackage{booktabs}
\usepackage{epstopdf}
\usepackage[htt]{hyphenat}
\usepackage{amsmath}
\usepackage{float}
\usepackage{indentfirst}
\usepackage{array}
\usepackage{enumitem}
\usepackage{bm}

\begin{document}
\begin{center}
\includegraphics[width=1.0\textwidth]{fit.pdf} % змініть розмір при необхідності

    \vspace{0.5cm} 
\Huge
\textsc{Vysoké učení technické v~Brně\\
}Fakulta informačních technologií\\
\vspace{\stretch{0.382}}
\Huge Technická správa \\
\LARGE T5: Zábavní průmysl \\
\LARGE  Model parku Walt Disney Studio v Disneylandu v Paříži\\
\vspace{\stretch{0.309}}

\Large 

\vspace{\stretch{0.309}}

\end{center}
{\Large \today \hfill
\begin{tabular}{l}
    Maryna Kucher, xkuche01 \\
    Vladyslava Bilyk, xbilyk03
\end{tabular}}
\thispagestyle{empty}

\newpage
\tableofcontents
\newpage

\section{Úvod}
V rámci tohoto projektu byla vytvořena modelová simulace jednoho z parků v Disneylandu v Paříži. Park nese název Walt Disney Studio a obsahuje 10 atrakcí a N tematických sekcí. Náš model ukazuje vstup do parku, volbu atrakce, čekání ve frontě na atrakci a samotnou jízdu na ní.

Cílem naší simulace bylo porovnat zisk, který park generuje od návštěvníků za současného nastavení (tj. cena za vstupenku na celý park), s variantou, kdy návštěvníci platí za vstup na každou atrakci zvlášť. Smyslem experimentů je demonstrovat, že pokud by Disneyland vybíral poplatek za každou atrakci zvlášť, místo jednorázového vstupného do samotného parku, dosáhl by vyššího zisku.

\subsection{Autoři a zdroje}
Tuto práci zpracovali Vladyslava Bilyk a Maryna Kucher. Informace byly získány z osobní návštěvy parku a byly podpořeny statistickými údaji z článků a internetových stránek.

\subsection{Validace}
Validace našeho programu byla provedena následujícím způsobem: experimentálně, po několika spuštěních modelu, byly statistické výstupy dat porovnány s očekávanými údaji, které odpovídají realitě. Každá fáze modelování byla analyzována z hlediska souladu s reálnými procesy. Vstupní parametry pro simulaci, jako je cena za vstupenku, počet návštěvníků za den, počet atrakcí, jejich kapacita a některé další údaje, byly čerpány z oficiálních stránek Disneylandu.

\section{Rozbor tématu a použitých metod/technologií}

\section{Koncepce}
\subsection{Koncepce - modelářská témata}
Model měla za cíl simulovat chování v zábavním parku Walt Disney Studio Park v Paříži a spočítat finanční výhodu přechodu na systém plateb za jednotlivé atrakce. Pro srovnání jsme použili cenu jednodenního nedatovaného vstupného ve výši 135 eur, jak je uvedeno na oficiálních stránkách [1]. Počet návštěvníků za den jsme převzali z veřejně dostupných statistických údajů za roky 2024/23 [2]. Průměrně park navštíví 5000 lidí denně, přičemž počet návštěvníků se může lišit v závislosti na sezóně. Na základě zdrojů, které ukazují relativní vytížení parku v konkrétní dny [3], jsme zjistili, že nejvíce návštěvníků je v létě, o něco méně na jaře, běžný počet na podzim a nejmenší vytížení parku je v zimě.

Otevírací doba parku se liší v závislosti na dni v týdnu a sezóně, proto jsme použili průměrné hodnoty od 9:00 do 21:00. Většina návštěvníků přichází během prvních hodin po otevření parku a čekají ve frontě na vstup. Předpokládáme, že si návštěvníci zakoupili vstupenky předem online a při vstupu do parku je stačí pouze naskenovat. Poté si návštěvník vybere, na kterou atrakci půjde. Na základě osobní zkušenosti a recenzí lidí ve veřejně dostupných zdrojích jsme zjistili, že volba atrakce závisí na čtyřech kritériích: zda již návštěvník atrakci navštívil, popularita atrakce, délka fronty u vstupu a vzdálenost atrakce.

Část atrakcí má také omezení podle výšky, proto jsme zavedli dve role návštěvníků: dospělé a děti. Před začátkem výběru atrakcí je každému uživateli přiřazena role. Vyšli jsme ze statistických údajů, podle kterých v Disneylandu tvoří 78,7\% návštěvníků dospělí a pouze 21,3\% děti.[4]

Disneyland parky mají uživatelsky přívětivý systém pro sledování front v reálném čase, na který jsme se snažili co nejvíce navázat. Pro výpočet čekací doby bylo nutné získat informace o délce trvání jednotlivých atrakcí a jejich kapacitě. Na základě dat získaných z fotografií a videí o parku jsme sestavili následující tabulku.
\section*{Tabulka atrakcí}

\begin{table}[h!]
	\centering
	\begin{tabular}{|l|c|c|c|c|c|c|c|}
		\hline
		\textbf{Název} &\textbf{Popularita}&\textbf{Single ride}&\textbf{Kapacita}&\textbf{V řadu}&\textbf{Doba(m)}&\textbf{Cena(€)}\\ \hline
		Flying Carpets        & Oblíbená         & Ne               & 20               & 4            & 5         & 5        \\ \hline
		Turtle Coaster        & Velmi oblíbená   & Ano              & 20               & 4            & 5         & 21       \\ \hline
		Toy Soldier           & Oblíbená         & Ano              & 36               & 6            & 5         & 5        \\ \hline
		Dog ZigZag            & Běžná            & Ne               & 28               & 2            & 5         & 5        \\ \hline
		Ratatouille           & Velmi oblíbená   & Ano              & 24               & 4            & 7         & 14       \\ \hline
		RC Racer              & Oblíbená         & Ne               & 20               & 2            & 5         & 5        \\ \hline
		Cars Road Trip        & Běžná            & Ne               & 20               & 2            & 5         & 5        \\ \hline
		Tower Terror          & Velmi oblíbená   & Ne               & 21               & 2            & 5         & 14       \\ \hline
		Spider Man            & Oblíbená         & Ano              & 24               & 2            & 5         & 16       \\ \hline
		Avengers 1            & Oblíbená         & Ano              & 24               & 2            & 5         & 12       \\ \hline
	\end{tabular}
	\caption{Popis atrakcí parku}
	\label{tab:attractions}
\end{table}

Také jsme uvedli navrhovanou cenu za každou atrakci. Jako základ jsme použili cenu za prémiový přístup k atrakci (Disneyland nabízí možnost příplatku za přístup k atrakci bez čekání ve frontě. Protože se tato cena vypočítává na základě popularity a nákladnosti atrakce, rozhodli jsme se, že bude vhodná pro naši simulaci).

Samostatná jízda je další specifikou Disney parků. Princip spočívá v tom, že pokud chcete atrakci navštívit ve skupině/rodině, bez problémů vás usadí společně. Kvůli tomu ale atrakce nejsou vždy plně obsazené (například v jednom řádku chce sedět rodina o třech lidech a za nimi jde pár. Ti se nevejdou do čtyřmístného řádku, takže jedno místo zůstane neobsazené), což prodlužuje čekací dobu. Proto existuje fronta pro samostatnou jízdu, která umožňuje zaplnit tato volná místa v řadách.

Nejprve jsou tedy usazeni lidé ze standardní fronty a na volná místa pak lidé ze fronty pro samostatnou jízdu. Tím se snižuje počet neobsazených míst a zkracuje se čekací doba.
Bohužel neexistují přesné informace o procentu lidí, kteří využívají možnost single rider. Proto jsme se rozhodli, že v naší simulaci bude člověk nejprve porovnávat dobu čekání (dostupnou na mapách Disneylandu, které umožňují sledovat vytíženost atrakcí v reálném čase). Pokud bude fronta na single rider kratší, v 50\% případů si člověk vybere tuto možnost.


Významnou součástí zábavy v parcích Disneylandu je také procházka tematickými zónami. Určili jsme, že tato procházka může trvat od 40 do 60 minut, což je dostatek času na to, aby návštěvník prošel tematickou zónu a ochutnal jídlo z místních stánků.

Asi půl hodiny před zavřením parku lidé přestávají vybírat nové atrakce. Dojdou k poslední zvolené atrakci a vystojí tam frontu. V tuto dobu začínají atrakce fungovat na plný výkon, maximálně využívají svou kapacitu a snaží se přepravit všechny lidi, kteří ještě čekají ve frontě.



\subsection{Koncepce - implementační témata}

\section{Architektura simulačního modelu/simulátoru}

\section{Podstata simulačních experimentů a jejich průběh}

\section{Shrnutí simulačních experimentů a závěr}
\bibliographystyle{czechiso}
\bibliography{citations.bib} 

\end{document}

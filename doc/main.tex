
\documentclass[a4paper,12pt]{article}
\usepackage[czech,slovak]{babel}
\usepackage[a4paper, total={170mm, 240mm}, top=2cm, left=3cm, right=3cm]{geometry}
\usepackage[utf8]{inputenc}
\usepackage[IL2]{fontenc}
\usepackage{times}
\usepackage[hidelinks]{hyperref}
\usepackage{graphicx}
\usepackage{booktabs}
\usepackage{epstopdf}
\usepackage[htt]{hyphenat}
\usepackage{amsmath}
\usepackage{float}
\usepackage{indentfirst}
\usepackage{array}
\usepackage{enumitem}
\usepackage{bm}
\usepackage{amsmath}
\usepackage{algorithm}
\usepackage{algpseudocode}

\begin{document}
\begin{center}
\includegraphics[width=1.0\textwidth]{fit.pdf} % змініть розмір при необхідності

    \vspace{0.5cm} 
\Huge
\textsc{Vysoké učení technické v~Brně\\
}Fakulta informačních technologií\\
\vspace{\stretch{0.382}}
\Huge Technická správa \\
\LARGE T5: Zábavní průmysl \\
\LARGE  Model parku Walt Disney Studio v Disneylandu v Paříži\\
\vspace{\stretch{0.309}}

\Large 

\vspace{\stretch{0.309}}

\end{center}
{\Large \today \hfill
\begin{tabular}{l}
    Maryna Kucher, xkuche01 \\
    Vladyslava Bilyk, xbilyk03
\end{tabular}}
\thispagestyle{empty}

\newpage
\tableofcontents
\newpage

\section{Úvod}
V rámci tohoto projektu byla vytvořena modelová simulace jednoho z parků v Disneylandu v Paříži. Park nese název Walt Disney Studio a obsahuje 10 atrakcí a 4 tematických sekcí. Náš model ukazuje vstup do parku, volbu atrakce, čekání ve frontě na atrakci a samotnou jízdu na ní.

Cílem naší simulace bylo porovnat zisk, který park generuje od návštěvníků za současného nastavení (tj. cena za vstupenku na celý park), s variantou, kdy návštěvníci platí za vstup na každou atrakci zvlášť. Smyslem experimentů je demonstrovat, že pokud by Disneyland vybíral poplatek za každou atrakci zvlášť, místo jednorázového vstupného do samotného parku, dosáhl by vyššího zisku.

\subsection{Autoři a zdroje}
Tuto práci zpracovali Vladyslava Bilyk a Maryna Kucher. Informace byly získány z osobní návštěvy parku a byly podpořeny statistickými údaji z článků a internetových stránek.

\subsection{Validace}
Validace našeho programu byla provedena následujícím způsobem: experimentálně, po několika spuštěních modelu, byly statistické výstupy dat porovnány s očekávanými údaji, které odpovídají realitě. Každá fáze modelování byla analyzována z hlediska souladu s reálnými procesy. Vstupní parametry pro simulaci, jako je cena za vstupenku, počet návštěvníků za den, počet atrakcí, jejich kapacita a některé další údaje, byly čerpány z oficiálních stránek Disneylandu.

\section{Rozbor tématu a použitých metod/technologií}
Rozvrh parku se často mění, proto jsme pro experimenty vybrali jednu z jeho variant: od 9:00 do 21:00. Návštěvníci přicházejí v rozmezí od 10:00 do 12:00 a poté se řadí do fronty k turniketům u vchodu. Celkem je zde 30 turniketů. Park nabízí návštěvníkům dva typy zábavy: tematické parky a 10 atrakcí (1). Po výběru atrakce se návštěvník postaví do fronty podle svého výběru: buď do běžné fronty, určené pro skupiny návštěvníků, kteří chtějí sedět vedle sebe, nebo do fronty „Single Rider“, pokud ji daná atrakce má. Tato fronta slouží k obsazení volných míst po běžné frontě jedním návštěvníkem. Po dokončení jízdy na atrakci se návštěvník může rozhodnout, zda se projde parkem, nebo pokračuje v návštěvě dalších atrakcí. S blížící se dobou zavírání lidé přestávají chodit na atrakce a odcházejí buď do tematických parků na večerní program, nebo z Disneylandu.
\subsection{Popis použitých technologií a metod}
Pro náš projekt jsme použili knihovnu SIMLIB a programovací jazyk C++. C++ byl zvolen, protože tento jazyk zajišťuje snadnou práci s daty naší modelu, jako jsou fronty, atrakce a statistiky, což ho činí optimálním pro modelování. Pro inspiraci základním použitím procesů v SIMLIB jsme využili příklady z demonstračních cvičení (2).

\section{Koncepce}
\subsection{Koncepce - modelářská témata}
Model měla za cíl simulovat chování v zábavním parku Walt Disney Studio Park v Paříži a spočítat finanční výhodu přechodu na systém plateb za jednotlivé atrakce. Pro srovnání jsme použili cenu jednodenního nedatovaného vstupného ve výši 135 eur, jak je uvedeno na oficiálních stránkách [1]. Počet návštěvníků za den jsme převzali z veřejně dostupných statistických údajů za roky 2024/23 [2]. Průměrně park navštíví 5000 lidí denně, přičemž počet návštěvníků se může lišit v závislosti na sezóně. Na základě zdrojů, které ukazují relativní vytížení parku v konkrétní dny [3], jsme zjistili, že nejvíce návštěvníků je v létě, o něco méně na jaře, běžný počet na podzim a nejmenší vytížení parku je v zimě.

Otevírací doba parku se liší v závislosti na dni v týdnu a sezóně, proto jsme použili průměrné hodnoty od 9:00 do 21:00. Většina návštěvníků přichází během prvních hodin po otevření parku a čekají ve frontě na vstup. Předpokládáme, že si návštěvníci zakoupili vstupenky předem online a při vstupu do parku je stačí pouze naskenovat. Poté si návštěvník vybere, na kterou atrakci půjde. Na základě osobní zkušenosti a recenzí lidí ve veřejně dostupných zdrojích jsme zjistili, že volba atrakce závisí na čtyřech kritériích: zda již návštěvník atrakci navštívil, popularita atrakce, délka fronty u vstupu a vzdálenost atrakce.

Část atrakcí má také omezení podle výšky, proto jsme zavedli dve role návštěvníků: dospělé a děti. Před začátkem výběru atrakcí je každému uživateli přiřazena role. Vyšli jsme ze statistických údajů, podle kterých v Disneylandu tvoří 78,7\% návštěvníků dospělí a pouze 21,3\% děti.[4]

Disneyland parky mají uživatelsky přívětivý systém pro sledování front v reálném čase, na který jsme se snažili co nejvíce navázat. Pro výpočet čekací doby bylo nutné získat informace o délce trvání jednotlivých atrakcí a jejich kapacitě. Na základě dat získaných z fotografií a videí o parku jsme sestavili následující tabulku.
\section*{Tabulka atrakcí}

\begin{table}[h!]
	\centering
	\begin{tabular}{|l|c|c|c|c|c|c|c|}
		\hline
		\textbf{Název} &\textbf{Popularita}&\textbf{Single ride}&\textbf{Kapacita}&\textbf{V řadu}&\textbf{Doba(m)}&\textbf{Cena(€)}\\ \hline
		Flying Carpets        & Oblíbená         & Ne               & 20               & 4            & 5         & 5        \\ \hline
		Turtle Coaster        & Velmi oblíbená   & Ano              & 20               & 4            & 5         & 21       \\ \hline
		Toy Soldier           & Oblíbená         & Ano              & 36               & 6            & 5         & 5        \\ \hline
		Dog ZigZag            & Běžná            & Ne               & 28               & 2            & 5         & 5        \\ \hline
		Ratatouille           & Velmi oblíbená   & Ano              & 24               & 4            & 7         & 14       \\ \hline
		RC Racer              & Oblíbená         & Ne               & 20               & 4            & 5         & 5        \\ \hline
		Cars Road Trip        & Běžná            & Ne               & 20               & 2            & 5         & 5        \\ \hline
		Tower Terror          & Velmi oblíbená   & Ne               & 21               & 21           & 5         & 14       \\ \hline
		Spider Man            & Oblíbená         & Ano              & 20               & 4            & 5         & 16       \\ \hline
		Avengers 1            & Oblíbená         & Ano              & 24               & 2            & 5         & 12       \\ \hline
	\end{tabular}
	\caption{Popis atrakcí parku}
	\label{tab:attractions}
\end{table}

Také jsme uvedli navrhovanou cenu za každou atrakci. Jako základ jsme použili cenu za prémiový přístup k atrakci (Disneyland nabízí možnost příplatku za přístup k atrakci bez čekání ve frontě. Protože se tato cena vypočítává na základě popularity a nákladnosti atrakce, rozhodli jsme se, že bude vhodná pro naši simulaci).

Samostatná jízda je další specifikou Disney parků. Princip spočívá v tom, že pokud chcete atrakci navštívit ve skupině/rodině, bez problémů vás usadí společně. Kvůli tomu ale atrakce nejsou vždy plně obsazené. Například tříčlenná rodina chce sedět v jedné řadě a za ní jdou dva lidé, kteří chtějí sedět také spolu. Pokud jsou čtyři sedadla v řadě, budou sedět ve dvou řadách: tři v jedné řadě a dvě ve druhé. Pak budou tři místa volná, což prodlužuje čekací dobu a atrakce není plně využita. Proto existuje fronta pro samostatnou jízdu, která umožňuje zaplnit tato volná místa v řadách.

Nejprve jsou tedy usazeni lidé ze standardní fronty a na volná místa pak lidé ze fronty pro samostatnou jízdu. Tím se snižuje počet neobsazených míst a zkracuje se čekací doba.
Bohužel neexistují přesné informace o procentu lidí, kteří využívají možnost single rider. Proto jsme se rozhodli, že v naší simulaci bude člověk nejprve porovnávat dobu čekání (dostupnou na mapách Disneylandu, které umožňují sledovat vytíženost atrakcí v reálném čase). Pokud bude fronta na single rider kratší, v 50\% případů si člověk vybere tuto možnost.


Významnou součástí zábavy v parcích Disneylandu je také procházka tematickými zónami. Určili jsme, že tato procházka může trvat od 40 do 60 minut, což je dostatek času na to, aby návštěvník prošel tematickou zónu a ochutnal jídlo z místních stánků.

Asi půl hodiny před zavřením parku lidé přestávají vybírat nové atrakce. Dojdou k poslední zvolené atrakci a vystojí tam frontu. V tuto dobu začínají atrakce fungovat na plný výkon, maximálně využívají svou kapacitu a snaží se přepravit všechny lidi, kteří ještě čekají ve frontě.



\subsection{Koncepce - implementační témata}
Model práce zábavního parku je omezena na jeden den jeho provozu.
Systém se skládá z následujících hlavních procesů: návštěvníci a atrakce. Každá atrakce má následující vstupní parametry: popularita, vhodnost pro děti, dostupnost služby Single Rider, kapacita, délka trvání atrakce a její cena. Generátor okamžitě vytvoří všechny návštěvníky, kteří postupně v průběhu ranního časového intervalu přicházejí ke vstupu do zábavního parku. Atrakce se aktivují, když do fronty vstoupí návštěvníci, kteří naplní kapacitu atrakce, čímž začíná proces nástupu a následné jízdy.

Důležitý proces uzavírání atrakcí se aktivuje 10 minut před koncem provozní doby a umožní posledním návštěvníkům využít atrakce. Významnou součástí procesů jsou fronty: fronta na vstup a fronty na atrakce. Obsluha front probíhá na principu: kdo přijde první, je první obsloužen.

\begin{algorithm}[H]
\caption{Algoritmus každého návštěvníka}
\begin{algorithmic}[1]
    \State Postaví se do fronty u vstupních turniketů a poté projde jedním z nich.
    \While{čas \textless  uzavírací čas}
        \State Vybere si atrakci.
        \State Postaví se do fronty k atrakci a čeká, až se naplní požadovaný počet lidí.
        \State Nástup na atrakci a následná jízda.
        \If{chce jít do tematického parku}
            \State Procházka tematickým parkem.
        \EndIf
    \EndWhile
\end{algorithmic}
\end{algorithm}

\textbf{Popis procesu atrakce}
\begin{enumerate}[label=\textbf{\arabic*.}]
    \item Atrakce čeká, až se naplní dostatečný počet lidí.
    \item Probíhá nástup návštěvníků podle řad.
    \item Nejprve se obsadí místa ve frontě \textit{Regular}, počet se volí od 1 osoby až po kapacitu jedné řady; zbývající místa jsou dle možností obsazena návštěvníky z fronty \textit{Single Rider}.
    \item Přechod k další řadě, a tak dále, dokud není atrakce plně obsazena.
    \item Provedení jízdy na atrakci.
    \item Aktivace procesů návštěvníků, konkrétně jejich odchod z atrakce.
\end{enumerate}

\section{Architektura simulačního modelu/simulátoru}
V simulátoru jsou návštěvníci parku implementováni procesem `Person`. Tato třída má následující atributy:  

\begin{itemize}
    \item \texttt{currentAttraction}: ID atrakce, na které se návštěvník právě nachází (nebo -1, pokud je u vstupu).  
    \item \texttt{current\_attraction}: struktura obsahující podrobné informace o atrakci, na které se návštěvník právě nachází nebo kterou plánuje navštívit.  
    \item Proměnná \texttt{distanceToNext}: pomáhá vypočítat přibližný čas, který návštěvník potřebuje na přesun k další atrakci.  
    \item Pole \texttt{visitedAttractions}: uchovává informace o atrakcích, které návštěvník již navštívil.  
\end{itemize}

Proces atrakce je implementován třídou `Ride`, která obsahuje:  

\begin{itemize}
    \item Funkci pro běžné chování atrakce.  
    \item Funkci \texttt{closingSoon()}, která umožňuje atrakci spustit pro poslední návštěvníky ve frontě, aniž by byla zcela naplněna.  
\end{itemize} 

Atributy této třídy jsou: 

\begin{itemize}
    \item Strukturu \texttt{current\_attraction} s údaji o konkrétní atrakci.  
    \item Ukazatel na frontu \texttt{SingleRideQ} (pro jednotlivé jezdce).  
    \item Ukazatel na běžnou frontu \texttt{RegularRideQ}.  
\end{itemize} 


Kompilace programu probíhá příkazem \texttt{make}.  
První experiment se spustí příkazem \texttt{make run}.  
Analogickým příkazem je také \texttt{./model summer}.  

Příklady možného spuštění programu:  
\begin{itemize}
    \item \texttt{./model summer [+/-číslo]}: spustí simulaci parku během letního dne s obvyklým počtem návštěvníků pro toto období. Nepovinný parametr `[+/- číslo]` upraví cenu všech atrakcí: zvýší nebo sníží základní cenu. Aktuální ceny atrakcí jsou nastaveny na úroveň z `Premier Access`. Tento parametr umožňuje testovat různé cenové změny.
    \item \texttt{./model winter [+/-číslo]} 
    \item \texttt{./model spring [+/-číslo]}
    \item \texttt{./model autumn [+/-číslo]}
\end{itemize} 
Analogicky lze simulovat zimní, jarní a podzimní dny.  



\section{Podstata simulačních experimentů a jejich průběh}

Hlavním cílem našeho projektu bylo zjistit, zda bude finančně výhodné (a do jaké míry) přejít u parku Disney Walt Studios z modelu jednorázové vstupenky, která platí na všechny atrakce po celý den, na model, kde návštěvník platí jednotlivě za každou atrakci.

\subsection{Experiment 1}

\subsubsection{První část}
První část experimentu se spouští pomocí příkazu \texttt{make run} nebo \texttt{./model summer} a simuluje jeden den provozu parku.

Pro simulaci jsme použili data z období nejvyšší návštěvnosti (léto) a na jejich základě vymodelovali zatížení parku v těchto podmínkách. Výstupy simulace zahrnují detailní analýzu, která obsahuje:
\begin{itemize}
	\item Maximální a průměrnou délku front u jednotlivých atrakcí.
	\item Průměrný čas strávený návštěvníkem ve frontách během dne.
	\item Počet atrakcí navštívených jedním návštěvníkem za celý den.
	\item Celkový příjem z prodeje vstupenek na celý den.
	\item Celkový příjem získaný z individuálních plateb za jedotlivé atrakce.
\end{itemize}


Výsledky modelování ukazují, že v letním období lidé strávili téměř 4 hodiny denně neustále ve frontách, což je charakteristické pro Disneyland. Příjem vzrostl o 168,58\%.

\begin{table}[h!]
	\centering
	\caption{Přehled atrakcí 1–5: délky front a popularita}
	\label{tab:attractions_overview_1_5}
	\begin{tabular}{|l|c|c|c|c|c|}
		\hline
		\textbf{Parametr}&\textbf{Flying Carpets}&\textbf{Turtle Coaster}&\textbf{Toy Soldier}&\textbf{Dog ZigZag}&\textbf{Ratatouille}\\ \hline
		Max. délka fronty   &161                 & 80                 & 68                 & 149                 & 203      \\ \hline
		Prům. délka fronty  &67                 & 37                 & 16                 & 48                 & 120         \\ \hline
		Popularita          &Populární         &Velmi populár.        &Populární          & Běžná          &Velmi populár.\\ \hline
	\end{tabular}
\end{table}
\begin{table}[h!]
	\centering
	\caption{Přehled atrakcí 6-10: délky front a popularita}
	\label{tab:attractions_overview_6_10}
	\begin{tabular}{|l|c|c|c|c|c|}
		\hline
		\textbf{Parametr}   & \textbf{RC  Racer}& \textbf{Cars Road Trip}& \textbf{Tower Terror}& \textbf{Spider Man}& \textbf{Avengers}\\ \hline
		Max. délka fronty   & 114                 & 120                & 172                 & 136               & 121               \\ \hline
		Prům. délka fronty  & 30                & 47                 & 40                & 72                 & 85               \\ \hline
		Popularita          & Populární             & Běžná              & Velmi populární        & Populární          & Populární   \\ \hline
	\end{tabular}
\end{table}


\begin{table}[h!]
	\centering
	\caption{Finanční a návštěvní údaje simulace Summer}
	\label{tab:financial_and_visit_data}
	\begin{tabular}{|l|c|}
		\hline
		\textbf{Parametr}                                         & \textbf{Hodnota} \\ \hline
		Příjem z denních vstupenek (EUR)                          & 1 012 500,00 €   \\ \hline
		Příjem z jednotlivých plateb za atrakce (EUR)             & 2 700 786,00 €   \\ \hline
		Průměrný čas strávený ve frontách během dne na osobu (min) & 239,39 minut     \\ \hline
		Průměrný počet navštívených atrakcí na osobu             & 27               \\ \hline
		\textbf{Procentuální nárůst příjmu}                       & 166,74\%         \\ \hline
	\end{tabular}
\end{table}

\subsubsection{Druhá část}
Druhá část tohoto experimentu spočívala v hodnocení chování modelu během méně vytížené zimní sezóny. Program je spuštěn následujícím způsobem: ./model winter.

Takže v zimním období vykazuje model dokonce vyšší procentní nárůst příjmu než v létě, konkrétně 180,69\%. To je očekávaný výsledek, protože počet návštěvníků je menší, zatížení parku nižší a lidé stíhají projet více atrakcí za den.

\begin{table}[h!]
	\centering
	\caption{Přehled atrakcí 1–5: délky front a popularita}
	\label{tab:attractions_overview_1_5}
	\begin{tabular}{|l|c|c|c|c|c|}
		\hline
		\textbf{Parametr}&\textbf{Flying Carpets}&\textbf{Turtle Coaster}&\textbf{Toy Soldier}&\textbf{Dog ZigZag}&\textbf{Ratatouille}\\ \hline
		Max. délka fronty   &135                 & 62                & 88                 & 106                 & 180     \\ \hline
		Prům. délka fronty  &26                 & 29                & 19                 & 23                 & 91         \\ \hline
		Popularita          &Populární         &Velmi populár.        &Populární          & Běžná          &Velmi populár.\\ \hline
	\end{tabular}
\end{table}
\begin{table}[h!]
	\centering
	\caption{Přehled atrakcí 6-10: délky front a popularita}
	\label{tab:attractions_overview_6_10}
	\begin{tabular}{|l|c|c|c|c|c|}
		\hline
		\textbf{Parametr}   & \textbf{RC  Racer}& \textbf{Cars Road Trip}& \textbf{Tower Terror}& \textbf{Spider Man}& \textbf{Avengers}\\ \hline
		Max. délka fronty   & 85                 & 112                & 169                 & 100               & 109               \\ \hline
		Prům. délka fronty  & 23                & 21                 & 44                & 33                 & 25               \\ \hline
		Popularita          & Populární             & Běžná              & Velmi populární        & Populární          & Populární   \\ \hline
	\end{tabular}
\end{table}

\begin{table}[h!]
	\centering
	\caption{Finanční a návštěvní údaje simulace Winter}
	\label{tab:financial_and_visit_data}
	\begin{tabular}{|l|c|}
		\hline
		\textbf{Parametr}                                         & \textbf{Hodnota} \\ \hline
		Příjem z denních vstupenek (EUR)                          & 607500,00 €   \\ \hline
		Příjem z jednotlivých plateb za atrakce (EUR)             & 1795644,00 €   \\ \hline
		Průměrný čas strávený ve frontách během dne na osobu (min) & 256,276 minut     \\ \hline
		Průměrný počet navštívených atrakcí na osobu             & 30               \\ \hline
		\textbf{Procentuální nárůst příjmu}                       & 195,53\%         \\ \hline
	\end{tabular}
\end{table}


Závěr: Příjem se zvýšil o 166,74\% v rušném období a o 195,53\%  v méně rušném období. To je očekávaný a pozitivní výsledek. Nicméně si uvědomujeme, že změna modelu platby za atrakce v Disneylandu může vést k poklesu počtu návštěvníků. Proto jsme se rozhodli provést druhý experiment, který umožňuje snížit ceny všech atrakcí pomocí zavedení dodatečného parametru. Považujeme, že takové snížení cen nejen pozitivně ovlivní zájem návštěvníků, ale také zajistí požadovaný růst příjmu.

\subsection{Experiment 2}
První část experimentu se spouští pomocí příkazu \texttt{./model summer -4}. V tomto experimentu spustíme simulaci jednoho dne v létě, ale snížíme cenu každé atrakce o 4 eura, abychom přiblížili výsledky realitě. Příliš vysoké ceny by totiž mohly odradit návštěvníky od vyzkoušení některé z atrakcí.

\begin{table}[h!]
	\centering
	\caption{Finanční a návštěvní údaje simulace Summer}
	\label{tab:financial_and_visit_data}
	\begin{tabular}{|l|c|}
		\hline
		\textbf{Parametr}                                         & \textbf{Hodnota} \\ \hline
		Příjem z denních vstupenek (EUR)                          & 1 012 500,00 €   \\ \hline
		Příjem z jednotlivých plateb za atrakce (EUR)             & 1 862 414,00 €   \\ \hline
		Průměrný čas strávený ve frontách během dne na osobu (min) & 239,39 minut     \\ \hline
		Průměrný počet navštívených atrakcí na osobu             & 27               \\ \hline
		\textbf{Procentuální nárůst příjmu}                       & 83,95\%         \\ \hline
	\end{tabular}
\end{table}

Závěr: Experiment ukazuje, že i při výrazném snížení cen je možné dosáhnout značného zvýšení zisku. Při správném nastavení ceny vstupenky bude možné nejen udržet zájem návštěvníků, ale také zvýšit zisk. Tato strategie je užitečná nejen pro park z finančního hlediska, ale i pro návštěvníky. Ti získají bezplatný přístup do mnoha tematických zón a trhů se zbožím a produkty Disneylandu, stejně jako možnost platit pouze za ty atrakce, které si přejí navštívit.

\section{Shrnutí simulačních experimentů a závěr}
Simulace tak poskytuje komplexní přehled o provozu parku v nejvytíženějším období a umožňuje srovnání výnosnosti obou platebních modelů. Z výsledků experimentů vyplývá, že rozdělení plateb mezi jednotlivé atrakce, místo vybírání jednorázového poplatku za vstupenku, výrazně zvýší zisk, i když budou ceny atrakcí nižší, než je uvedeno v Disneylandu jako Premier Access. A to bez zbytečné zátěže na atrakce a zaměstnance, což jsme ověřili analýzou maximálního počtu lidí ve frontách na každou atrakci a průměrného počtu lidí ve frontách. Výsledky těchto dat byly očekávané a odpovídaly realitě.

\bibliographystyle{czechiso}
\bibliography{citations.bib} 

\end{document}

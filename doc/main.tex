
\documentclass[a4paper,12pt]{article}
\usepackage[czech,slovak]{babel}
\usepackage[a4paper, total={170mm, 240mm}, top=2cm, left=3cm, right=3cm]{geometry}
\usepackage[utf8]{inputenc}
\usepackage[IL2]{fontenc}
\usepackage{times}
\usepackage[hidelinks]{hyperref}
\usepackage{graphicx}
\usepackage{booktabs}
\usepackage{epstopdf}
\usepackage[htt]{hyphenat}
\usepackage{amsmath}
\usepackage{float}
\usepackage{indentfirst}
\usepackage{array}
\usepackage{enumitem}
\usepackage{bm}
\usepackage{amsmath}
\usepackage{algorithm}
\usepackage{algpseudocode}

\begin{document}
\begin{center}
\includegraphics[width=1.0\textwidth]{fit.pdf} % змініть розмір при необхідності

    \vspace{0.5cm} 
\Huge
\textsc{Vysoké učení technické v~Brně\\
}Fakulta informačních technologií\\
\vspace{\stretch{0.382}}
\Huge Technická správa \\
\LARGE T5: Zábavní průmysl \\
\LARGE  Model parku Walt Disney Studio v Disneylandu v Paříži\\
\vspace{\stretch{0.309}}

\Large 

\vspace{\stretch{0.309}}

\end{center}
{\Large \today \hfill
\begin{tabular}{l}
    Maryna Kucher, xkuche01 \\
    Vladyslava Bilyk, xbilyk03
\end{tabular}}
\thispagestyle{empty}

\newpage
\tableofcontents
\newpage

\section{Úvod}
V rámci tohoto projektu byla vytvořena modelová simulace jednoho z parků v Disneylandu v Paříži. Park nese název Walt Disney Studio a obsahuje 10 atrakcí a N tematických sekcí. Náš model ukazuje vstup do parku, volbu atrakce, čekání ve frontě na atrakci a samotnou jízdu na ní.

Cílem naší simulace bylo porovnat zisk, který park generuje od návštěvníků za současného nastavení (tj. cena za vstupenku na celý park), s variantou, kdy návštěvníci platí za vstup na každou atrakci zvlášť. Smyslem experimentů je demonstrovat, že pokud by Disneyland vybíral poplatek za každou atrakci zvlášť, místo jednorázového vstupného do samotného parku, dosáhl by vyššího zisku.

\subsection{Autoři a zdroje}
Tuto práci zpracovali Vladyslava Bilyk a Maryna Kucher. Informace byly získány z osobní návštěvy parku a byly podpořeny statistickými údaji z článků a internetových stránek.

\subsection{Validace}
Validace našeho programu byla provedena následujícím způsobem: experimentálně, po několika spuštěních modelu, byly statistické výstupy dat porovnány s očekávanými údaji, které odpovídají realitě. Každá fáze modelování byla analyzována z hlediska souladu s reálnými procesy. Vstupní parametry pro simulaci, jako je cena za vstupenku, počet návštěvníků za den, počet atrakcí, jejich kapacita a některé další údaje, byly čerpány z oficiálních stránek Disneylandu.

\section{Rozbor tématu a použitých metod/technologií}
Rozvrh parku se často mění, proto jsme pro experimenty vybrali jednu z jeho variant: od 9:00 do 21:00. Návštěvníci přicházejí v rozmezí od 10:00 do 12:00 a poté se řadí do fronty k turniketům u vchodu. Celkem je zde 30 turniketů. Park nabízí návštěvníkům dva typy zábavy: tematické parky a 10 atrakcí (1). Po výběru atrakce se návštěvník postaví do fronty podle svého výběru: buď do běžné fronty, určené pro skupiny návštěvníků, kteří chtějí sedět vedle sebe, nebo do fronty „Single Rider“, pokud ji daná atrakce má. Tato fronta slouží k obsazení volných míst po běžné frontě jedním návštěvníkem. Po dokončení jízdy na atrakci se návštěvník může rozhodnout, zda se projde parkem, nebo pokračuje v návštěvě dalších atrakcí. S blížící se dobou zavírání lidé přestávají chodit na atrakce a odcházejí buď do tematických parků na večerní program, nebo z Disneylandu.
\subsection{Popis použitých technologií a metod}
Pro náš projekt jsme použili knihovnu SIMLIB a programovací jazyk C++. C++ byl zvolen, protože tento jazyk zajišťuje snadnou práci s daty naší modelu, jako jsou fronty, atrakce a statistiky, což ho činí optimálním pro modelování. Pro inspiraci základním použitím procesů v SIMLIB jsme využili příklady z demonstračních cvičení (2).

\section{Koncepce}
\subsection{Koncepce - modelářská témata}
\subsection{Koncepce - implementační témata}
Model práce zábavního parku je omezena na jeden den jeho provozu.
Systém se skládá z následujících hlavních procesů: návštěvníci a atrakce. Každá atrakce má následující vstupní parametry: popularita, vhodnost pro děti, dostupnost služby Single Rider, kapacita, délka trvání atrakce a její cena. Generátor okamžitě vytvoří všechny návštěvníky, kteří postupně v průběhu ranního časového intervalu přicházejí ke vstupu do zábavního parku. Atrakce se aktivují, když do fronty vstoupí návštěvníci, kteří naplní kapacitu atrakce, čímž začíná proces nástupu a následné jízdy.

Důležitý proces uzavírání atrakcí se aktivuje 10 minut před koncem provozní doby a umožní posledním návštěvníkům využít atrakce. Významnou součástí procesů jsou fronty: fronta na vstup a fronty na atrakce. Obsluha front probíhá na principu: kdo přijde první, je první obsloužen.

\begin{algorithm}
\caption{Algoritmus každého návštěvníka}
\begin{algorithmic}[1]
    \State Postaví se do fronty u vstupních turniketů a poté projde jedním z nich.
    \While{čas \textless  uzavírací čas}
        \State Vybere si atrakci.
        \State Postaví se do fronty k atrakci a čeká, až se naplní požadovaný počet lidí.
        \State Nástup na atrakci a následná jízda.
        \If{chce jít do tematického parku}
            \State Procházka tematickým parkem.
        \EndIf
    \EndWhile
\end{algorithmic}
\end{algorithm}

\textbf{Popis procesu atrakce}
\begin{enumerate}[label=\textbf{\arabic*.}]
    \item Atrakce čeká, až se naplní dostatečný počet lidí.
    \item Probíhá nástup návštěvníků podle řad.
    \item Nejprve se obsadí místa ve frontě \textit{Regular}, od 1 osoby až po kapacitu jedné řady; zbývající místa jsou dle možností obsazena návštěvníky z fronty \textit{Single Rider}.
    \item Přechod k další řadě, a tak dále, dokud není atrakce plně obsazena.
    \item Provedení jízdy na atrakci.
    \item Aktivace procesů návštěvníků, konkrétně jejich odchod z atrakce.
\end{enumerate}

\section{Architektura simulačního modelu/simulátoru}
V simulátoru jsou návštěvníci parku implementováni procesem `Person`. Tato třída má následující atributy:  

\begin{itemize}
    \item \texttt{currentAttraction}: ID atrakce, na které se návštěvník právě nachází (nebo -1, pokud je u vstupu).  
    \item \texttt{current\_attraction}: struktura obsahující podrobné informace o atrakci, na které se návštěvník právě nachází nebo kterou plánuje navštívit.  
    \item Proměnná \texttt{distanceToNext}: pomáhá vypočítat přibližný čas, který návštěvník potřebuje na přesun k další atrakci.  
    \item Pole \texttt{visitedAttractions}: uchovává informace o atrakcích, které návštěvník již navštívil.  
\end{itemize}

Proces atrakce je implementován třídou `Ride`, která obsahuje:  

\begin{enumerate}
    \item Funkci pro běžné chování atrakce.  
    \item Funkci \texttt{closingSoon()}, která umožňuje atrakci spustit pro poslední návštěvníky ve frontě, aniž by byla zcela naplněna.  
\end{enumerate} 

Atributy této třídy jsou: 

\begin{itemize}
    \item Strukturu \texttt{current\_attraction} s údaji o konkrétní atrakci.  
    \item Ukazatel na frontu \texttt{SingleRideQ} (pro jednotlivé jezdce).  
    \item Ukazatel na běžnou frontu \texttt{RegularRideQ}.  
\end{itemize} 


Kompilace programu probíhá příkazem \texttt{make}.  
První experiment se spustí příkazem \texttt{make run}.  
Analogickým příkazem je také \texttt{./model summer}.  

Příklady možného spuštění programu:  
\begin{itemize}
    \item \texttt{./model summer [+/- číslo]}: spustí simulaci parku během letního dne s obvyklým počtem návštěvníků pro toto období. Nepovinný parametr `[+/- číslo]` upraví cenu všech atrakcí: zvýší nebo sníží základní cenu. Aktuální ceny atrakcí jsou nastaveny na úroveň z `Premier Access`. Tento parametr umožňuje testovat různé cenové změny.
    \item \texttt{./model winter [+/- číslo]} 
    \item \texttt{./model spring [+/- číslo]}
    \item \texttt{./model autumn [+/- číslo]}
\end{itemize} 
Analogicky lze simulovat zimní, jarní a podzimní dny.  



\section{Podstata simulačních experimentů a jejich průběh}

\section{Shrnutí simulačních experimentů a závěr}
\bibliographystyle{czechiso}
\bibliography{citations.bib} 

\end{document}
